\section{Background Theory}
\subsection{Petri Net}
According to \citeA{murata1989}, a Petri net is defined as a tuple $N = (P, T, F, M_0)$, where $P$ is a finite set of places, $T$ is a finite set of transitions, and $F$ represents the flow relation, 
allowing us to visualize the causal dependencies between events. The state of a Petri net is defined by its \textbf{marking} $M$, 
which is a vector of non-negative integers of size $|P|$. The dynamic behavior is governed by the firing rule: a transition $t$ is enabled if $M(p) \ge 1$ for 
all input places $p \in t$. When $t$ fires, the new marking $M'$ is given by the state equation:
\begin{equation*}
    M' = M + C \cdot u
\end{equation*}
where $C$ is the incidence matrix and $u$ is the firing vector.

This assignment specifically focuses on \textbf{1-safe Petri nets}. A net is 1-safe if for every reachable marking $M \in Reach(M_0)$, and for every place $p \in P$, the number of tokens is at most 1:
\begin{equation*}
    \forall M \in Reach(M_0), \forall p \in P: M(p) \in \{0, 1\}
\end{equation*}
This property allows the global state to be encoded as a Boolean vector, facilitating the use of symbolic logic.

\subsection{Binary Decision Diagrams}
Binary Decision Diagrams (BDDs) are a canonical directed acyclic graph (DAG) 
data structure used to represent Boolean functions \citeA{bryant1986}. A Boolean 
function $f: \{0, 1\}^n \to \{0, 1\}$ is represented as a rooted graph where non-terminal 
nodes are labeled with variables $x_i$ and terminal nodes represent values 0 (False) and 1 (True).

The power of BDDs lies in their \textbf{canonical form} (Reduced Ordered BDD - ROBDD). For a fixed variable 
ordering, any Boolean function has a unique BDD representation. This allows for constant-time equivalence 
checking and efficient logical operations with complexity proportional to the graph size rather than the 
exponential state space.

In the context of this assignment, we utilize BDDs to solve the state space explosion problem inherent in concurrent systems.
Since the Petri net is \textbf{1-safe} \cite{cheng1995}, the number of tokens in any place $p_i$ is strictly bounded to $\{0, 1\}$. Thus, a global marking $M$ can be encoded as a Boolean vector $(x_1, x_2, \dots, x_{|P|})$, where $x_i = 1$ if and only if place $p_i$ contains a token. The set of all reachable markings is represented by a characteristic function $\chi_{Reach}$.

\subsection{Integer Linear Programming}
Integer Linear Programming (ILP) is a mathematical optimization technique where the 
objective is to minimize or maximize a linear function subject to linear equality and 
inequality constraints, with the variables restricted to integers. The standard form is:
\begin{equation*}
    \text{Maximize } c^T x \quad \text{subject to } Ax \le b, \ x \in \mathbb{Z}^n
\end{equation*}
In Petri nets, ILP is often used for structural analysis because the fundamental state equation 
($M = M_0 + A \cdot \sigma$, where $A$ is the incidence matrix) is a linear system. 

This assignment requires a hybrid approach, combining the "exact" state space from BDDs with the "structural" equations of ILP.
For \textbf{Deadlock Detection (Task 4)}, we formulate linear constraints representing ``dead'' states 
(where all transitions are disabled) and intersect this condition with the BDD-computed reachability set to 
identify genuine reachable deadlocks. For \textbf{Optimization (Task 5)}, the objective to maximize $c^\top M$ 
is addressed by directing the search for the optimal marking within the valid solution space encoded by the BDD, 
ensuring the result is both optimal and reachable within the 1-safe Petri net.