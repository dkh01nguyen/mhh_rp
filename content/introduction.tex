\section{Introduction}

\subsection{Motivation}
Petri nets are among the most fundamental and elegant mathematical models for describing concurrent, 
distributed, and event-driven systems. Since their introduction in the early 1960s by Carl Adam Petri, they have 
become a cornerstone of formal methods and system verification. They provide a rigorous graphical 
and formal way to represent the interaction between conditions (places) and events (transitions) 
through the flow of tokens. This framework is ideal for modeling systems such as manufacturing 
lines, communication protocols, and complex regulatory networks.

However, the analysis of these systems is often computationally challenging due to the 
\textbf{state space explosion problem}. The concurrency inherent in complex systems can lead to 
an exponential number of reachable markings (states), making traditional explicit enumeration 
methods - like Breadth-First Search or Depth-First Search - infeasible for all but small systems.

\subsection{The Integration of Symbolic and Algebraic Reasoning}
This assignment explores the integration of two powerful computational techniques: symbolic and 
algebraic reasoning.

\begin{itemize}
    \item \textbf{Symbolic Representation (Binary Decision Diagrams - BDDs):} We leverage BDDs to compactly encode the potentially 
        vast set of reachable markings. By representing the state space symbolically using canonical Boolean functions, BDDs enable 
        efficient memory management and faster formal verification compared to explicit state storage.
    \item \textbf{Algebraic Optimization (Integer Linear Programming - ILP):} We utilize ILP as a flexible, optimization-based framework 
        to reason about non-local properties of the Petri net. Core questions related to structural properties, such as invariant analysis 
        and deadlock detection, can be formulated and solved as systems of linear inequalities based on the net's incidence matrix.
\end{itemize}

\subsection{Project Objectives}
The central goal of this assignment is to foster both theoretical insight and hands-on skills in bridging abstract models with 
algorithmic analysis. We shall build an application integrating these computational ideas to analyze \textbf{1-safe Petri nets}. 
The specific contributions of this report are organized around the following key tasks:

\begin{enumerate}
    \item \textbf{Parsing and Explicit Reachability:} Implementing a parser for the PNML standard to construct the model and explicitly enumerating states for performance baseline.
    \item \textbf{Symbolic Reachability Analysis:} Using BDDs to symbolically construct the set of reachable markings and comparing its performance against the explicit approach.
    \item \textbf{Deadlock Detection:} Applying ILP formulations in combination with the BDD representation to detect deadlocks.
    \item \textbf{Optimization over Reachable Markings:} Solving a final optimization problem, maximizing a linear objective function over the computed set of reachable markings.
\end{enumerate}

\subsection{Synthesis}
By integrating the structural modeling strength of Petri Nets, the memory efficiency of BDDs, and the analytical power of ILP, this project provides an
    approach necessary for tackling complex engineering problems in formal verification and optimization.